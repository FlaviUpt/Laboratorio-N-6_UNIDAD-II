\section{Actividad No 01 – Comandos} 
¿Que sucede al ejecutar los siguientes comandos? 

\begin{itemize}
	\item STARTUP OPEN
	\\Abre los ficheros. Es en este modo cuando podemos decir que la base de datos está
		completamente operativa.
	\\Cuando arrancamos este estado existe la posibilidad de arrancar una misma base de datos
    con distintas instancias, indicando que fichero init.ora queremos que utilice.
	

	\item STARTUP MOUNT
	\\Analiza que los ficheros que le indica el parámetro CONTROLFILE en el archivo de
		parámetros, son los que los que se han puesto y que la ruta indicada sea la correcta.
	\\Este estado se usa en caso de que queramos hacer una copia de seguridad de la base de
		datos.
	\\Como en el estado anterior, en modo MOUNT pueden aparecer algunos problemas como los
		siguientes:
		
		\begin{itemize}
			\item Problemas de hardware.
			\item Que existan ficheros no sincronizados (lo cual implicaría que se hiciera un 	recover)
			\item No existan los ficheros de datos o de redo log que el CONTROLFILE debe leer. 
			Si el fichero de datos que falta no es crítico, puedo arrancar sin él y después recuperarlo 
			con un backup.
			
		\end{itemize}


\end{itemize} 